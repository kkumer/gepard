%% $Id$

\documentclass[12pt]{article}

\usepackage{a4wide}

%%%%%%%%%%%%%%%%%%%%%%%%%%%%%%%%%%%%%%%%%%%%%%%%%%%%%
%% Macros for comments
\usepackage{amssymb}
\newcounter{comment}

\newcommand{\comminline}[1]{{%
\refstepcounter{comment}%
\ttfamily\small[$\blacksquare$ \textbf{\underline{Comment}
$\sharp$\thecomment:} #1]}}

\newenvironment{commblock}%
{\refstepcounter{comment}%
\begin{quote}\renewcommand{\baselinestretch}{1}
\ttfamily\small$\blacksquare$ \textbf{\underline{Comment} $\sharp$\thecomment:}}%
{\end{quote}}


\newcommand{\replline}[1]{{
%\refstepcounter{comment}%
\ttfamily\small[$\blacktriangleright$ \textbf{\underline{Reply}
$\sharp$\thecomment:} #1]}}

\newenvironment{replblock}%
{%\refstepcounter{comment}
\begin{quote}\renewcommand{\baselinestretch}{1}
\ttfamily\small$\blacktriangleright$ \textbf{\underline{Reply} $\sharp$\thecomment:}}%
{\end{quote}}
%%%%%%%%%%%%%%%%%%%%%%%%%%%%%%%%%%%%%%%%%%%%%%%%%%%%%

\begin{document}

\title{\texttt{gepard}: calculating Compton form factors from moments of GEneral PARton Distributions } 
\author{Version: 0.9b2}
\date{\today}
\maketitle

\begin{abstract}
Fortran package \texttt{gepard} calculates singlet Compton form factor
$\mathcal{H}$ and corresponding Compton cross-sections, as well as
DIS form factor $F_2$ by evolving GPDs from a given ansatz shape.  Together
with \texttt{Minuit} minimization subroutine GPD parameters can be fitted to
small-$\xi$ experimental DVCS and DIS data. For installation instructions and
basic usage, see README file coming with the package.
\end{abstract}

\section{Formula for $\mathcal{H}$}  

\begin{eqnarray*}
{{\cal H}^{\rm S}}(\xi,\Delta^2,{\cal Q}^2)
&\!\!\!=\!\!\!& \frac{1}{2i}\int_{c-i \infty}^{c+ i \infty}\!
dj\,\xi^{-j-1} \left[i + \tan \left(\frac{\pi j}{2}\right) \right]
\mbox{\boldmath $C$}_{j}({\cal Q}^2/\mu^2,\alpha_s(\mu)) 
\mbox{\boldmath $H$}_{j} (\xi,\Delta^2,\mu^2) \\[2ex]
&\!\!\!=\!\!\!& \frac{(\xi/2)^{-c-1}}{\Gamma(3/2)}\Big\{i \Im\! m\: e^{i\phi} 
\int_{0}^{\infty} dy  (\xi/2)^{- y e^{i\phi}} 
\frac{\Gamma(5/2 + c + y e^{i\phi})}{\Gamma(3 + c + y e^{i\phi})}
\texttt{FPW}(j) \\[2ex]
& & + \Im\! m\:  e^{i\phi} 
\int_{0}^{\infty} dy  (\xi/2)^{- y e^{i\phi}}
\frac{\Gamma(5/2 + c + y e^{i\phi})}{\Gamma(3 + c + y e^{i\phi})}
\tan\big(\frac{\pi(c + y e^{i\phi})}{2}\big)
\texttt{FPW}(j)
\label{<Hformula>}
\end{eqnarray*} 
where $j=c+ye^{i\phi}$, and Schwartz reflection principle is
used.
The usual textbook Mellin-Barnes contour is obtained by
using $\phi = \pi/2$, but, say, $\phi = 3\pi /4$ gives faster
convergence. Parameter of Mellin-Barnes contour are specified
via parameters \texttt{C} and \texttt{PHI} in \texttt{init.f},
and by default they are equal to 0.5 and $3\pi/4$, respectively.

DVCS Wilson coefficients $\mbox{\boldmath $C$}_{j}({\cal Q}^2/\mu^2,\alpha_s(\mu))$
are given in Eq. (18) of \cite{Kumericki:2006xx}. 

\section{Formula for $F_2$}  

\begin{eqnarray*}
{{F_2}^{\rm S}}(\xi,{\cal Q}^2)
&\!\!\!=\!\!\!& \frac{Q_{S}^2}{2 \pi i}\int_{c-i \infty}^{c+ i \infty}\!
dj\,\xi^{-j}
\mbox{\boldmath $C$}^{\rm DIS}_{2,j}(Q^2/\mu^2,\alpha_s(\mu))
\mbox{\boldmath $H$}_{j} (\xi=0,\Delta^2=0,\mu^2) \\[2ex]
&\!\!\!=\!\!\!& \frac{Q_{S}^2}{\pi} \xi^{-c}
\int_{0}^{\infty} dy\: \Im\! m\: e^{i\phi} \xi^{- y e^{i\phi}}
\texttt{FPW}(j)
\label{<F2formula>}
\end{eqnarray*}

DIS Wilson coefficients $\mbox{\boldmath $C$}^{\rm DIS}_{2,j}(Q^2/\mu^2,\alpha_s(\mu))$
are given e.g. in \cite{vanNeerven:2000uj}. They are equal to those in
 Eq. (18) of \cite{Kumericki:2006xx} without prefactor in front of square brackets there.

\section{Numerical Mellin-Barnes Integration}  
\label{sect:Integration}
Numerical integration is performed in a way very similar to the one described in
\cite{Vogt:2004ns}: Integrand is largest and
fluctuates most rapidly around zero, mostly due to the factor $\xi^{-j}$. As we
move away from real axis, both
amplitude (for $\phi > \pi/2$) and oscillation frequency decrease exponentially.
Thus we divide integration region in exponentially expanding intervals
defined by points.

\begin{equation}
 \{0, 0.01, 0.025, 0.067, 0.18, 0.5, 1.3, 3.7, 10\}
\label{<intervals>}
\end{equation}

Between each two points standard 8-point Gaussian integration is employed, according
to Eq. (25.4.30) from \cite{AbS}.

This is for \texttt{SPEED}=1. Otherwise integration intervals are defined by using
only points with indices

\begin{equation}
 \{ 1, 1 + \texttt{SPEED}, 1 + 2 \texttt{SPEED}, \ldots \}
\label{<speedintervals>}
\end{equation}

from the above list. Thus for \texttt{SPEED}=1,2, and 4 we use in total 64, 32, and 16
integration points, respectively.

\section{Initialization files}

Execution of basic gepard routines depends on initialization parameters specified in
file \texttt{GEPARD.INI}, and Minuit fitting depends on additional two initialization files
\texttt{FIT.INI} and \texttt{MINUIT.CMD}, as well as experimental data files, specified
in  \texttt{FIT.INI}. The contents of these files is explained in following subsections.

\textbf{NB:} Character constants (i.e. strings) in these files should be enclosed in
single quotation marks, as required by Fortran 77 standard!

\subsection{\texttt{GEPARD.INI}}

File has structure:

\begin{verbatim}
2          SPEED        1(most accurate), 2 or 4(fastest)
1          P            N^{P}LO order (0, 1 or 2)
3          NF           number of active flavours
0.05d0     AS0          value of astrong/(2 Pi) at 2.5 GeV
1.D0       RF2          ratio {\cal Q}^2/{\mu_{f}^2}
1.D0       RR2          ratio {\cal Q}^2/{\mu_{r}^2}
'CSBAR'    SCHEME       
'FIT'      ANSATZ       GPD ansatz (see parwav.f subroutine HJ)
\end{verbatim}

Only the first column is read by the program and it contains
values of Fortran variables specified in second column and described
in third one. Columns two and three are just convenient reminders for the user. 

\begin{itemize}
\item
\texttt{SPEED} --- Controls the speed and accuracy of the execution. Values can
presently be only 1, 2, and 4, where 1 is slowest, and most accurate case.
This is described in more detail in section \ref{sect:Integration}.

\item
\texttt{P} --- Perturbation theory approximation level. 0 = LO, 1 = NLO, and
  2 = NNLO.

\item
\texttt{NF} --- Number of active quark flavours

\item
\texttt{AS0} --- Value of $\alpha_{\rm s}(2.5 \textrm{GeV})/(2 \pi)$. For various
test, as well as in \cite{Kumericki:2006xx} value 0.05 vas used. More realistic
value is 0.0432, which corresponds to $\alpha(M_Z) = 0.117$ \comminline{Check this!}

\item
\texttt{RF2} --- Ratio of photon momentum and factorization scales squared
 $ {\cal Q}^2/{\mu_{f}^2}$

\item
\texttt{RR2} --- Ratio of photon momentum and factorization scales squared
 $ {\cal Q}^2/{\mu_{r}^2}$

\item
\texttt{SCHEME} --- Renormalization scheme. Can presently be \texttt{'CSBAR'} or
\texttt{'MSBAR'} corresponding to $\bar{CS}$ and $\bar{MS}$ schemes, respectively,
but note that evolution is at the moment implemented only for \texttt{'CSBAR'} scheme.

\item
\texttt{ANSATZ} --- Name of ansatz for conformal moments of GPD on input scale. It
should correspond to a  \texttt{IF\ldots ELSE IF \ldots END IF} stanza in subroutine
\texttt{HJ} in file \texttt{parwav.f}. By default, implemented ansaetze are
\texttt{TOY} (used in some old development notebooks), \texttt{SOFT} and \texttt{HARD},
corresponding to ansaetze from \cite{Kumericki:2006xx}, and \texttt{FIT}, corresponding
to ansatz with six free parameters used for fitting, see description of 
\texttt{MINUIT.CMD} below.

\end{itemize}

%\bibliographystyle{h-physrev4}
\bibliographystyle{$HOME/tex/inputs/JHEP-2}
%\bibliographystyle{mynatbib}
\bibliography{$HOME/Lit/kkumer}



\end{document}


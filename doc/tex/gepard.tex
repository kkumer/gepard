%% $Id$

\documentclass[12pt]{article}

\usepackage{a4wide}


\begin{document}

\title{\texttt{gepard}: calculating Compton form factors from moments of General PARton Distributions } 
\author{Version: 0.89}
\date{\today}
\maketitle

\begin{abstract}
Fortran package \texttt{gepard} calculates singlet Compton form factor
$\mathcal{H}$ and corresponding Compton cross-sections, as well as
DIS form factor $F_2$ by evolving GPDs from a given ansatz shape.  Together
with \texttt{minuit} minimization subroutine GPD parameters can be fitted to
small-$\xi$ experimental DVCS and DIS data. For installation instructions and
basic usage, see README file coming with the package.
\end{abstract}

\section{Formula for $\mathcal{H}$}  

\begin{eqnarray*}
{{\cal H}^{\rm S}}(\xi,\Delta^2,{\cal Q}^2)
&\!\!\!=\!\!\!& \frac{1}{2i}\int_{c-i \infty}^{c+ i \infty}\!
dj\,\xi^{-j-1} \left[i + \tan \left(\frac{\pi j}{2}\right) \right]
\mbox{\boldmath $C$}_{j}({\cal Q}^2/\mu^2,\alpha_s(\mu)) 
\mbox{\boldmath $H$}_{j} (\xi,\Delta^2,\mu^2) \\[2ex]
&\!\!\!=\!\!\!& \frac{(\xi/2)^{-c-1}}{\Gamma(3/2)}\Big\{i \Im\! m\: e^{i\phi} 
\int_{0}^{\infty} dy  (\xi/2)^{- y e^{i\phi}} 
\frac{\Gamma(5/2 + c + y e^{i\phi})}{\Gamma(3 + c + y e^{i\phi})}
\texttt{FPW}(j) \\[2ex]
& & + \Im\! m\:  e^{i\phi} 
\int_{0}^{\infty} dy  (\xi/2)^{- y e^{i\phi}}
\frac{\Gamma(5/2 + c + y e^{i\phi})}{\Gamma(3 + c + y e^{i\phi})}
\tan\big(\frac{\pi(c + y e^{i\phi})}{2}\big)
\texttt{FPW}(j)
\label{<Hformula>}
\end{eqnarray*} 
where $j=c+ye^{i\phi}$, and Schwartz reflection principle is
used.
The usual textbook Mellin-Barnes contour is obtained by
using $\phi = \pi/2$, but, say, $\phi = 3\pi /4$ gives faster
convergence. Parameter of Mellin-Barnes contour are specified
via parameters \texttt{C} and \texttt{PHI} in \texttt{INIT.DAT}.

DVCS Wilson coefficients $\mbox{\boldmath $C$}_{j}({\cal Q}^2/\mu^2,\alpha_s(\mu))$
are given in Eq. (18) of \cite{Kumericki:2006xx}. 

\section{Formula for $F_2$}  

\begin{eqnarray*}
{{F_2}^{\rm S}}(\xi,{\cal Q}^2)
&\!\!\!=\!\!\!& \frac{Q_{S}^2}{2 \pi i}\int_{c-i \infty}^{c+ i \infty}\!
dj\,\xi^{-j}
\mbox{\boldmath $C$}^{\rm DIS}_{2,j}(Q^2/\mu^2,\alpha_s(\mu))
\mbox{\boldmath $H$}_{j} (\xi=0,\Delta^2=0,\mu^2) \\[2ex]
&\!\!\!=\!\!\!& \frac{Q_{S}^2}{\pi} \xi^{-c}
\int_{0}^{\infty} dy\: \Im\! m\: e^{i\phi} \xi^{- y e^{i\phi}}
\texttt{FPW}(j)
\label{<F2formula>}
\end{eqnarray*}

DIS Wilson coefficients $\mbox{\boldmath $C$}^{\rm DIS}_{2,j}(Q^2/\mu^2,\alpha_s(\mu))$
are given e.g. in \cite{vanNeerven:2000uj}. They are equal to those in
 Eq. (18) of \cite{Kumericki:2006xx} without prefactor in front of square brackets there.

\section{Numerical Mellin-Barnes Integration}  

Numerical integration is performed in a way very similar to the one described in
\cite{Vogt:2004ns}: Integrand is largest and
fluctuates most rapidly around zero, mostly due to the factor $\xi^-j$. As we
move away from real axis, both
amplitude (for $\phi > \pi/2$) and oscillation frequency decrease exponentially.
Thus we divide integration region in exponentially expanding intervals
defined by points.

\begin{equation}
 \{0, 0.01, 0.025, 0.067, 0.18, 0.5, 1.3, 3.7, 10\}
\label{<intervals>}
\end{equation}

Between each two points standard 8-point Gaussian integration is employed, according
to Eq. (25.4.30) from \cite{AbS}.

This is for \texttt{SPEED}=1. Otherwise integration intervals are defined by using
only points with indices

\begin{equation}
 \{ 1, 1 + \texttt{SPEED}, 1 + 2 \texttt{SPEED}, \ldots \}
\label{<speedintervals>}
\end{equation}

from the above list. Thus for \texttt{SPEED}=1,2, and 4 we use in total 64, 32, and 16
integration points, respectively.

%\bibliographystyle{h-physrev4}
\bibliographystyle{$HOME/tex/inputs/JHEP-2}
%\bibliographystyle{mynatbib}
\bibliography{$HOME/Lit/kkumer}


\end{document}

